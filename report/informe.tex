\documentclass[11pt, a4paper]{article}

\usepackage[spanish]{babel}
\usepackage[utf8]{inputenc}
\usepackage{graphicx}

% Margin set to a4wide
\usepackage{geometry}
\usepackage{layout}

\geometry{
  left=2.5cm,
  right=2.5cm,
  top=3.5cm,
  bottom=3cm
}

\usepackage{amsmath}
\usepackage{amstext}
\usepackage{amssymb}
\usepackage{hyperref}

\linespread{1.3}

\title{
    \large{Computación Paralela: Trabajo Final}\\
    \huge{Similitud Coseno Punto a Punto}
}

\author{Cristian A. Cardellino}

\date{12 de Agosto de 2015}

\begin{document}
  \maketitle

  \section{Introducción}

  El presente informe describe el trabajo final realizado para la cátedra de
  posgrado ``Computación Paralela''. El proyecto elegido fue la paralelización
  del algoritmo de {\em similitud coseno punto a punto} (pairwise cosine
  similarity) de una matriz.
  
  A grandes rasgos, el algoritmo toma una matriz y compara cada fila de esta
  contra todas las demás filas utilizando similitud coseno entre los vectores
  conformados por las filas. Luego devuelve una matriz de distancia con las
  distancias de cada par de filas en la matriz.

  La similitud coseno es una medida de la similitud existente entre dos
  vectores, en un espacio vectorial que posee producto punto, con el que se
  evalúa el valor del coseno del ángulo comprendido entre ellos. Esta medida
  proporciona un valor igual a 1 si el ángulo es cero, es decir ambos vectores
  apuntan al mismo lugar. Ante cualquier ángulo existente entre los vectores la
  medida arrojaría un valor menor a uno. En caso de vectores ortogonales la
  similitud es nula. Finalmente, si los vectores son opuestos, la medida sería
  -1.

  La similitud coseno puede ser aplicada a varias dimensiones, y es más
  comúnmente utilizada en espacios de alta dimensionalidad, e.g. búsqueda y
  recuperación de información (information retrieval) o minería de textos (text
  mining). 

  La similitud punto a punto es utilizada como técnica en sistemas de
  recomendación (recommender systems), particularmente en filtrado colaborativo
  (collaborative filtering), donde una matriz representa valoraciones
  (ratings) que ciertos usuarios les dan a ciertos items y esto a su vez es
  utilizado para recomendar dichos items a otros usuarios basándose en la
  similitud que tienen valorando items.

  La paralelización del presente problema se da en que la similitud entre dos
  filas de la matriz es completamente independiente por cada par de filas que
  haya en la matriz. No obstante, realizar este trabajo secuencialmente es del
  orden $\mathcal{O}(n^{2})$, donde $n$ es la cantidad de filas de la matriz.

  % TODO: Revisar outline
%  El informe se estructura de la siguiente manera: en la
%  Sección~\ref{sec:problema} se describe el problema particular sobre el que se
%  trabajaron las técnicas de optimización, se presentan los datos utilizados y
%  la manera en que se preprocesaron y como se representan dentro del problema;
%  la Sección~\ref{sec:experimentos} describe la configuración de los
%  experimentos, detalla el hardware utilizado en los mismos, se describe en
%  mayor detalle las distintas mejoras realizadas y los problemas con los
%  encontrados al realizarlas y se presentan las métricas y mediciones
%  utilizadas; la Sección~\ref{sec:resultados} muestra los resultados de los
%  experimentos y hace un análisis general de los mismos; finalmente el reporte
%  concluye en la Sección~\ref{sec:conclusiones}, donde se hace una observación
%  general del problema y los objetivos alcanzados.
 
  \section{Descripción del problema, datos y representación}\label{sec:problema}

  \subsection{Filtrado colaborativo basado en items}

  Los sistemas de recomendación (recommender systems) buscan predecir las
  preferencias de ciertos usuarios sobre ciertos items. En los últimos años se
  ha visto que estos sistemas se volvieron extremadamente comunes en distintas
  áreas como libros, películas, música, y productos en general. Grandes
  jugadores del medio del e-commerce o que brindan servicios de streaming hacen
  uso de estos sistemas para brindar una mejor experiencia de usuario (Netflix,
  Amazon, Spotify, entre otros).

  Dentro de estos sistemas, una técnica muy utilizada es el filtrado
  colaborativo (collaborative filtering) cuya idea parte de que usuarios
  similares tendrán gustos similares. El filtrado colaborativo es un método que
  busca hacer predicciones automáticas (filtrado) sobre los intereses de un
  usuario recolectando las preferencias de varios usuarios. La suposición que
  hace esta técnica es que si un usuario {\it A} tiene una opinión similar a un
  usuario {\it B} sobre determinado asunto, {\it A} es más probable que tenga
  una opinión similar a {\it B} en un asunto diferente a que tenga una opinión
  similar a algún otro usuario aleatorio {\it C}. Esto también es conocido como
  {\em filtrado colaborativo basado en usuarios} (user-based collaborative
  filtering) y tiene algunos problemas como:

  \begin{itemize}
    \item Bajo desempeño cuando hay muchos items pero pocas calificaciones.
    \item La cantidad de usuarios suele sobrepasar por varias magnitudes al
        número de items.
    \item Las preferencias de los usuarios pueden cambiar, lo que implica tener
        que recalcular todo el sistema.
  \end{itemize}

  Una variante de la técnica de filtrado colaborativo llamada {\em filtrado
  colaborativo basado en items}~\cite{linden2001collaborative} (item-based
  collaborative filtering) fue patentada por Amazon, y resuelve la mayoría de
  los problemas presentados por el filtrado basado en usuarios, particularmente
  en sistemas donde hay más cantidad de usuarios que de items. La idea base de
  esta técnica es recomendar items que sean similares a otros items que el
  usuario ya calificó positivamente, midiendo la similitud de dichos items. En
  este proyecto se busca paralelizar y optimizar el código básico que se usa
  para hacer este filtrado colaborativo basado en items.

  \subsection{Datos utilizados}

  Para el desarrollo del proyecto se optó por datos estándar en la
  investigación de algoritmos de recomendación:
  MovieLens~\cite{Harper:2015:MDH:2866565.2827872}, de GroupLens Research. Este
  consta de distintas valoraciones (ratings) que usuarios del sitio web
  MovieLens\footnote{\url{http://movielens.org}} hicieron sobre distintas
  películas.

  GroupLens liberó distintos tamaños de su conjunto de datos para
  investigación. En este trabajo se hace uso de 4 de ellos:

  \begin{description} 
      \item[MovieLens 100k Dataset (ML100k)] Consta de 100000 valoraciones
          realizadas por 943 usuarios sobre 1682 películas.
      \item[MovieLens 1M Dataset (ML1M)] Consta de 1000209 valoraciones
          realizadas por 6040 usuarios sobre 3962 películas.
      \item[MovieLens 10M Dataset (ML10M)] Consta de 10000054 valoraciones
          realizadas por 69878 usuarios sobre 10677 películas.
      \item[MovieLens 20M Dataset (ML20M)] Consta de 20000263 valoraciones
          realizadas por 138493 usuarios sobre 26744 películas.
  \end{description}

  Las valoraciones de los usuarios están hechas en una escala de 5 estrellas
  (i.e. del rango [1, 5]), que en el caso de los datos de ML10M y ML20M pueden
  tener incrementos de media estrella (i.e., en el rango [0.5, 5.0]). Todos los
  conjuntos establecen que cada usuario valoró un mínimo de 20 películas.

  Los conjuntos ML100k y ML1M no tuvieron que ser modificados debido a que
  estos identificaron a sus películas y usuarios unívocamente con la cantidad.
  En el caso de los conjuntos ML10M y ML20M, estos tenían a sus usuarios y
  películas identificados con más números que la cantidad, por lo que se los
  preprocesó proyectando los valores de manera que estos fueran unívocos a la
  cantidad de usuarios/películas.

  \subsection{Representación de los datos}

  Se requiere de dos archivos para hacer funcionar el código: la matriz de
  valoraciones y la matriz de similitud para corrección. La primera se obtuvo a
  partir de los datos brindados por MovieLens, la segunda se calculó utilizando
  la librería de aprendizaje automático de Python ``Scikit
  Learn''~\cite{scikit-learn}.

  La matriz de valoraciones contiene una fila por cada película y una columna
  por cada usuario del conjunto de datos.  Cada celda de la matriz representa
  la cantidad de estrellas que el usuario le dio a la película, siendo este
  valor igual a cero cuando el usuario no hizo valoración alguna sobre dicha
  película.

  Debido a que el conjunto de datos establece que cada usuario valoró un mínimo
  de 20 películas, dejando la mayoría de las películas sin valorar, la matriz
  final de películas/usuarios es extremadamente rala, llegando a ser los
  valores no nulos menor al 1\% de la totalidad de la matriz.

  En los primeros experimentos la matriz películas/usuarios era representada
  por una matriz densa. Sin embargo, era solo viable para los casos de los
  conjuntos ML100k y ML1M que se podía cargar dicha matriz en memoria sin que
  afectara gravemente el desempeño de los algoritmos encargados de calcular la
  similitud coseno. Buscando mejorar este desempeño para conjuntos de datos de
  mayor tamaño se optó por el uso de representaciones ralas para matrices; y
  dado que se necesita un acceso rápido a las filas para su comparación se hace
  uso del formato Yale (o {\em compressed row storage}) para representar las
  matrices.
  
  El archivo con el que se guarda la matriz de valoraciones sigue la base del
  formato de intercambio {\em Matrix Market}~\cite{Boisvert:aa}. Se usa el
  formato COO de matrices ralas coordinadas que durante la etapa de carga se
  transforma en formato Yale. 

  La matriz de corrección tiene la particularidad de ser una {\em matriz de
  distancia}, lo que quiere decir que es simétrica a través de la diagonal
  principal. Por lo tanto es suficiente con guardar la mitad triangular
  superior de la matriz. Para esto se hace uso de la propuesta de James D.
  McCaffrey: ``Converting a Triangular Matrix to an
  Array''\footnote{\url{https://jamesmccaffrey.wordpress.com/2010/05/14/converting-a-triangular-matrix-to-an-array/}}.

  \section{Configuración de experimentos}\label{sec:experimentos}

  \subsection{Recursos de hardware}

  % TODO: Describir recursos de hardware luego de tener los experimentos
 
  \subsection{Algoritmos}

  \subsubsection{Algoritmo Base}

  \paragraph{Mejoras}

  \subsection{Métricas}

  \section{Resultados y análisis}\label{sec:resultados}

  % Para 20M no se puede levantar la matriz

  \section{Conclusiones}\label{sec:conclusiones}

  \clearpage
  \bibliographystyle{abbrv}
  \bibliography{/Users/crscardellino/Documents/Posgrado/Bibliography/bibliography.bib}
\end{document}
